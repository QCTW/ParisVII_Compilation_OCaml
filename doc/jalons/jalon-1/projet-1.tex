%; whizzy paragraph
\documentclass[a4paper,8pt]{article}
\usepackage[francais]{babel}
\usepackage{fancybox}
%% Encodage et police
\usepackage{hyperref}
\usepackage{ucs}
\usepackage[utf8x]{inputenc}
\usepackage{lmodern}
\usepackage[T1]{fontenc}
\usepackage{geometry}
\usepackage{xcolor}
\geometry{hmargin=0.7cm, vmargin=2.5cm}
%% BNF
\usepackage{tabularx}
\usepackage{array}
\def\colsep@widearray{\,}
\newcolumntype{C}{@{\colsep@widearray}>{\({}}c<{{}\)}@{\colsep@widearray}}
\newcolumntype{S}{@{\colsep@widearray}>{{}}r<{{}}@{\colsep@widearray}}
\newcolumntype{L}{@{\colsep@widearray}>{{}}l<{{}}@{\colsep@widearray}}
\newcolumntype{R}{@{\colsep@widearray}>{\sl{}}r<{{}}@{\colsep@widearray}}
\newcolumntype{Z}{@{\colsep@widearray}>{{}}X<{{}}@{\colsep@widearray}}
\newcommand{\comment}[1]{\hfill \mbox{\textit{#1}}}
\newenvironment{BNF}[1][\linewidth]%
{\quote\tabularx{#1}{RSZ}\relax}%
{\endtabularx\endquote}
\newcommand{\kwd}[1]{\texttt{#1}}
\newcommand{\lex}[1]{\textsf{#1}}
\newcommand{\rul}[1]{\textsl{#1}}
\newcommand{\car}[1]{\texttt{#1}}
\newcommand{\plus}{$^+$ }
\newcommand{\mult}{$^*$ }
\newcommand{\opt}{$^*$ }
\newcommand{\meta}[1]{\textcolor{gray}{#1}}
\newcommand{\repeatseq}[1]{\textsl{\meta{\{}} #1 \textsl{\meta{\}}}}
\newcommand{\repeatseqplus}[1]{\textsl{\meta{\{}} #1 \textsl{\meta{\}}}⁺}
\newcommand{\optw}[1]{\textsl{\meta{[}} #1 \textsl{\meta{]}}}
\newcommand{\optseq}[1]{\optw{#1}}

\newlength\codewidth
\setlength\codewidth{15cm}
\newenvironment{code}[1][\codewidth]{
\begin{center}
\Sbox
\hspace{0.3cm}\minipage{#1}\small
}{
\endminipage
\endSbox\fbox{\TheSbox}
\end{center}
}

%%
\title{
\vspace{-1.5cm}
Projet du cours «~Compilation~» \\
Jalon 1~: Analyse lexicale et syntaxique de \textsc{Hopix}}
\date{\scriptsize version numéro \input{version}}

\begin{document}

\maketitle

\section{Spécification de la grammaire}

\subsection{Notations extra-lexicales}

Les commentaires, espaces, les tabulations et les sauts de ligne
jouent le rôle de séparateurs. Leur nombre entre les différents
symboles terminaux peut donc être arbitraire. Ils sont ignorés
par l'analyse lexicale: ce ne sont pas des lexèmes.

Les commentaires sont entourés des deux symboles «~\verb!{-!~» et
«~\verb!-}!~». Par ailleurs, ils peuvent être imbriqués.

On peut aussi introduire un commentaire à l'aide du symbole
«~\verb!--!~» : tout ce qui le suit jusqu'à la fin de la ligne
est interprété comme un commentaire.

\subsection{Symboles}

\paragraph{Symboles terminaux}
Les terminaux sont répartis en trois catégories : les mots-clés, les
identificateurs et la ponctuation.

Les mots-clés sont les noms réservés aux constructions du langage.
Ils seront écrits avec des caractères de machine à écrire
(comme par exemple les mots-clés \kwd{if} et \kwd{while}).

Les identificateurs sont constitués des identificateurs de variables,
d'étiquettes, de constructeurs de données et de types ainsi que des
littéraux, comprenant les constantes entières, les
caractères et les chaînes de caractères.  Ils seront écrits dans une
police \textsf{sans-serif} (comme par exemple \lex{type\_con} ou
\lex{int}). La classification des identificateurs est définie par les
expressions rationnelles suivantes~:

\begin{code}
\begin{BNF}
\rm \lex{alien\_infix\_id} & ≡ & \car{`} [A-Z a-z 0-9 \car{+} \car{-} \car{*} \car{/} \car{<} \car{=} \car{>} \car{\_}]$^+$ \car{`}
\comment{Identificateur d'opérateurs infixes}\\
\rm \lex{alien\_prefix\_id} & ≡ & \car{`} [A-Z a-z 0-9 \car{+} \car{-} \car{*} \car{/} \car{<} \car{=} \car{>} \car{\_}]$^+$
\comment{Identificateur d'opérateurs préfixes}\\
\rm \lex{var\_id}    & ≡ & [a-z] [A-Z a-z 0-9 \car{\_}]$^*$ | \lex{alien\_prefix\_id}
\comment{Identificateur de variables}\\
\rm \lex{constr\_id} & ≡ & [A-Z \car{\_}] [A-Z a-z 0-9 \car{\_}]$^*$
\comment{Identificateur de constructeurs de données}\\
\rm \lex{type\_con}   & ≡ & [a-z] [A-Z a-z 0-9 \car{\_}]$^*$
\comment{Identificateur de constructeurs de type}\\
\rm \lex{type\_variable}   & ≡ & \car{'} [a-z] [A-Z a-z 0-9 \car{\_}]$^*$
\comment{Identificateur de variables de type}\\
\rm \lex{int}        & ≡ &  [0-9]⁺ | 0[xX][0-9 a-f A-F]$^+$ | 0[Bb][0-1]$^+$ | 0[Oo][0-7]$^+$
\comment{Littéraux entiers}\\
\rm \lex{char}        & ≡ & '\lex{atom}'
\comment{Littéraux caractères}\\
\rm \lex{atom}       & ≡ & $\backslash$000 | \dots{} | $\backslash$255
| $\backslash$0[xX][0-9 a-f A-F]$^2$ | $\backslash$0[oO][0-7]+ | $\backslash$0[bB][0-1]+ | [\lex{printable}] \\
& | & \car{$\backslash\backslash$}
| \car{$\backslash${}'}
| \car{$\backslash$n}
| \car{$\backslash$t}
| \car{$\backslash$b}
| \car{$\backslash$r}
\\
\rm \lex{string}       & ≡ & " ( \lex{atom} $- \{$ \car{"} $\}$ | $\backslash$\car{"} )$^*$ "
\comment{Littéraux chaîne de caractères}\\

\end{BNF}
\end{code}

Autrement dit, les identificateurs de variables, de constructeurs de
type et de champs commencent par une lettre minuscule et peuvent
comporter ensuite des majuscules, des minuscules, des chiffres et le
caractère souligné~\car{\_}. Les identificateurs de constructeurs de
données peuvent comporter les mêmes caractères, mais doivent commencer
par une majuscule ou par un caractère~\car{\_}, tandis que les
variables de type commencent par un~caractère~\car{'}. Par ailleurs,
un identificateur d'opérateur est formé de symboles et de caractères
alphanumériques. S'il est entouré de deux~\car{`}, il est dit
\textit{infixe}. S'il est préfixé par un \car{`}, il est dit
\textit{préfixe}.

Les constantes entières sont constituées de chiffres en notation
décimale, en notation hexadécimale, en notation binaire ou en notation
octale. Les constantes entières sont prises dans
$[-2^{31}; 2^{31} - 1]$.

Les constantes de caractères sont décrites entre guillemets simples
(ce qui signifie en particulier que les guillemets simples doivent
être échappés dans les constantes de caractères). On y trouve en
particulier les symboles ASCII affichables. Par ailleurs, sont des
caractères valides~: les séquences d'échappement usuelles, ainsi que
les séquences d'échappement de trois chiffres décrivant le code ASCII
du caractère en notation décimale ou encore les séquences
d'échappement de deux chiffres décrivant le code ASCII en notation
hexadécimale.

Les constantes de chaîne de caractères sont formées d'une séquence de
caractères. Cette séquence est entourée de guillemets (ce qui signifie
en particulier que les guillemets doivent être échappés dans les
chaînes).

Les symboles seront notés avec la police "\car{machine à écrire}"
(comme par exemple~«~\car{(}~» ou ~«~\car{=}~»).

\paragraph{Symboles non-terminaux}
Les symboles non-terminaux seront notés à l'aide d'une police
légèrement inclinée (comme par exemple \rul{expr}).

Une séquence entre crochets est optionnelle (comme par exemple
«~\optw{\kwd{ref}}~»). Attention à ne pas confondre ces crochets
avec les symboles terminaux de ponctuation notés \car{[} et
  \car{]}. Une séquence entre accolades se répète zéro fois ou plus,
(comme par exemple
«~\car{(}~\rul{arg}~\repeatseq{\car{,}~\rul{arg}}~\car{)}~»).

\section{Grammaire en format BNF}

La grammaire du langage est spécifiée à l'aide du format BNF.

\paragraph{Programme} Un programme est constitué d'une séquence
de définitions de types et de valeurs.

\begin{code}[20cm]
\begin{BNF}
p  & $::=$  & \repeatseq{\rul{definition}} \comment{Programme} \\
\\
definition
& $::=$
& \kwd{type}
 \lex{type\_con} \optw{\car{(} \lex{type\_variable} \optw{\repeatseq{\car{,} \lex{type\_variable}}} \car{)}}
 \optw{\car{=} \rul{tdefinition}}
 \comment{Définition de type}
\\
& | &
\kwd{extern} \lex{var\_id} \car{:} \rul{type}
\comment{Valeurs externes}
\\
& | &  \rul{vdefinition}
\comment{Définition de valeur(s)}
\\
\\
tdefinition
& $::=$ &
\optw{\car{|}}
\lex{constr\_id}
\optw{\car{(} \rul{type} \repeatseq{\car{,} \rul{type}} \car{)}}
\comment{Type somme}
\\
&& \repeatseq{\car{|} \lex{constr\_id} \optw{\car{(} \rul{type} \repeatseq{\car{,} \rul{type}} \car{)}}}
\\
\\
vdefinition & $::=$ &
\kwd{val} \lex{var\_id} \optw{\car{:} \rul{type}}  \car{=} \rul{expr}
\comment{Valeur simple}
\\
& | & 
\kwd{fun}
\lex{var\_id}
\optw{\car{[} \rul{type\_variable} \repeatseq{\car{,} \rul{type\_variable}} \car{]}}
\car{(} \rul{pattern} \repeatseq{\car{,} \rul{pattern}} \car{)} \optw{\car{:} \rul{type}}  \car{=} \rul{expr}
\comment{Fonction}
\\
&&\repeatseq{\kwd{and} \lex{var\_id}
\optw{\car{[} \rul{type\_variable} \repeatseq{\car{,} \rul{type\_variable}} \car{]}}
\car{(} \rul{pattern} \repeatseq{\car{,} \rul{pattern}} \car{)} \optw{\car{:} \rul{type}}  \car{=} \rul{expr}}
\end{BNF}\smallskip
\end{code}

\paragraph{Types de données}
\noindent La syntaxe des types est donnée par la grammaire suivante:

\begin{code}
\begin{BNF}
type
& $::=$ & \lex{type\_con} \optw{\car{(} \rul{type} \repeatseq{\car{,} \rul{type}} \car{)}} \\
& | & \rul{type} \car{->} \rul{type} \\
& | & \lex{type\_variable} \\
& | & \car{(} \rul{type} \car{)}
\end{BNF}\smallskip
\end{code}

\paragraph{Expression}

La syntaxe des expressions du langage est donnée par la grammaire suivante.
\begin{code}[18cm]
\begin{center}
\begin{BNF}
expr      & $::=$ & \lex{int} \comment{Entier} \\
          &   | & \lex{char} \comment{Caractère} \\
          &   | & \lex{string} \comment{Chaîne de caractères} \\
          &   | & \lex{var\_id}                            \comment{Variable} \\
          &   | & \lex{constr\_id}
                  \optw{\car{[} \rul{type} \repeatseq{\car{,} \rul{type}} \car{]}}
                  \optw{\car{(} \rul{expr} \repeatseq{\car{,} \rul{expr}} \car{)}}
                  \comment{Construction d'une donnée étiquetée} \\
          &   | & \car{(} \rul{expr} \car{:} \rul{type} \car{)} \comment{Annotation de type} \\
                   &   | & \rul{expr} \repeatseq{\car{;} \rul{expr}}
                   \comment{Séquencement} \\
          &   | & \rul{vdefinition} \car{;} \rul{expr}
                  \comment{Définition locale} \\
                  &   | & \rul{expr}
                  \optw{\car{[} \rul{type} \repeatseq{\car{,} \rul{type}} \car{]}}
                  \car{(} \rul{expr} \repeatseq{\car{,} \rul{expr}} \car{)} \comment{Application} \\
          &   | & \car{$\backslash$}
                  \optw{\car{[} \rul{type\_variable} \repeatseq{\car{,} \rul{type\_variable}} \car{]}}
                  \car{(} \rul{pattern} \repeatseq{\car{,} \rul{pattern}} \car{)} \car{=>} \rul{expr}
                  \comment{Fonction anonyme} \\
          &   | & \rul{expr} \rul{binop} \rul{expr} \comment{Opérations binaires}\\
          &   | & \rul{expr} \car{?} \rul{branches}  \comment{Analyse de motifs} \\
          &   | & \kwd{if} \rul{expr} \kwd{then} \rul{expr}
          \repeatseq{\kwd{elif} \rul{expr}}
          \optw{\kwd{else} \rul{expr}}
          \comment{Conditionnelle} \\
          &   | & \kwd{ref} \rul{expr} \comment{Allocation} \\
          &   | & \rul{expr} \car{:=} \rul{expr} \comment{Affectation} \\
          &   | & \car{!} \rul{expr} \comment{Lecture} \\
          &   | & \kwd{while} \rul{expr} \car{\{} \rul{expr} \car{\}} \comment{Boucle} \\
          &   | & \car{(} \rul{expr} \car{)} \comment{Parenthésage} \\
\\
binop       & $::=$ & \car{+} | \car{-} | \car{*} | \car{/} | \car{\&\&} | \car{||} |
\car{=} | \car{<=} | \car{>=} | \car{<} | \car{>} | \lex{alien\_infix\_id}
\comment{Opérateurs binaires}
\\\\
branches  & $::=$ & \optw{\car{|}} \rul{branch} \repeatseq{\car{|} \rul{branch}} \comment{Liste de cas} \\
& | & \car{\{} \optseq{\car{|}} \rul{branch} \repeatseq{\car{|} \rul{branch}} \car{\}} \comment{... avec des accolades}
\\\\
branch    & $::=$ & \rul{pattern} \car{=>} \rul{expr} \comment{Cas d'analyse} \\
\end{BNF}
\end{center}
\end{code}

\paragraph{Motifs}
\noindent Les motifs (\textsl{patterns} en anglais), utilisés par
l'analyse de motifs, ont la syntaxe suivante:

\begin{code}
\begin{center}
  \begin{BNF}
pattern & $::=$ & \lex{constr\_id} \comment{Etiquette} \\
          &   | & \lex{var\_id} \comment{Motif universel liant}\\
          &   | & \car{\_} \comment{Motif universel non liant}\\
          &   | & \car{(} \rul{pattern} \car{)} \comment{Parenthésage} \\
          &   | & \rul{pattern} \car{:} \rul{type} \comment{Annotation de type} \\
          &   | & \lex{int} \comment{Entier} \\
          &   | & \lex{char} \comment{Caractère} \\
          &   | & \lex{string} \comment{Chaîne de caractères} \\
           &  | & \lex{constr\_id} {\car{(} \rul{pattern} \repeatseq{\car{,} \rul{pattern}} \car{)}}
          \comment{Valeurs étiquetées}\\
          &   | & \rul{pattern} \car{|} \rul{pattern} \comment{Disjonction} \\
          &   | & \rul{pattern} \car{\&} \rul{pattern} \comment{Conjonction} \\
\end{BNF}
\end{center}
\end{code}

\paragraph{Remarques}

Notez bien que la grammaire spécifiée plus haut est ambiguë! Vous devez fixer
des priorités entre les différentes constructions ainsi que des associativités
aux différents opérateurs. \textit{In fine}, c'est la batterie de tests en
ligne qui vous permettra de valider vos choix. Cependant, il est fortement
conseillé de poser des questions sur la liste de diffusion du cours pour
obtenir des informations supplémentaires sur les règles de disambiguation
associées à cette grammaire.

\section{Code fourni}

Un squelette de code vous est fourni, il est disponible sur le dépot GIT du cours.
\begin{center}
\url{git@moule.informatique.univ-paris-diderot.fr:Yann/compilation-m1-2016.git}
\end{center}

Vous devez vous connecter sur le Gitlab disponible ici:
\begin{center}
\url{http://moule.informatique.univ-paris-diderot.fr:8080}
\end{center}

et vous créer un dépot par branchement (\textit{fork}) du projet \verb!compilation-m1-2016!.

L'arbre de sources contient des \verb!Makefile!s ainsi que des modules
\textsc{O'Caml} à compléter.

La commande \verb!make! produit un exécutable appelé \verb!flap!. On
doit pouvoir l'appeler avec un nom de fichier en argument. En cas
de réussite (de l'analyse syntaxique), le code de retour de
ce programme doit être $0$. Dans le cas d'un échec, le code
de retour doit être $1$.

\section{Travail à effectuer}

La première partie du projet est l'écriture de l'analyseur lexical et
de l'analyseur syntaxique spécifiés par la grammaire précédente.

La compilation s'effectue par la commande«~\verb!make!~».

Le projet est à rendre \textbf{avant le}~:

\begin{code}
\begin{center}
\large\textbf{22 novembre 2016 à 23h59}
\end{center}
\end{code}

Pour finir, vous devez vous assurer des points suivants~:
\begin{code}
\begin{itemize}
\item Le projet contenu dans cette archive \textbf{doit compiler}.
\item Vous devez \textbf{être les auteurs} de ce projet.
\item Il doit être rendu \textbf{à temps}.
\end{itemize}
\end{code}

Si l'un de ces points n'est pas respecté, la note de $0$ vous sera affectée.

\section{Log}

\input{changes}

\end{document}
